\documentclass[times, utf8, zavrsni]{fer}
\usepackage{booktabs}
\usepackage{pdfpages}
\usepackage{placeins}

\begin{document}

% TODO: Navedite broj rada.
\thesisnumber{874}

% TODO: Navedite naslov rada.
\title{Aplikacija za praćenje rangiranja sveučilišta prema Šangajskoj listi}

% TODO: Navedite vaše ime i prezime.
\author{Ivan Bilobrk}

\maketitle

% Ispis stranice s napomenom o umetanju izvornika rada. Uklonite naredbu \izvornik ako želite izbaciti tu stranicu.
\includepdf[pages=-]{zadatak.pdf}

% Dodavanje zahvale ili prazne stranice. Ako ne želite dodati zahvalu, naredbu ostavite radi prazne stranice.
\zahvala{}

\tableofcontents

\chapter{Uvod}
U svijetu postoji više od 25000 sveučilišta. Svako od njih trudi se imati što bolje predavače, kvalitetniju nastavu, puno znanstvenih radova, sudjelovanja na konferencijama,
objava u časopisima te drugih raznih uspjeha. Sveučilištima je teško uskladiti svoju organizaciju bez neke povratne informacije o svojim uspjesima. Upravo zbog toga napravljene su razne
rang liste koje svakom sveučilištu pridružuju neku numeričku vrijednost te na osnovu nje ih sortiraju. Na ovaj način svako sveučilište dobiva svoju poziciju 
koja može služiti kao mjerilo uspjeha, ukazati na potencijalne probleme na sveučilištu te tako omogućiti uvođenje promjena na sveučilištu kako bi 
na sljedećoj rang listi sveučilište bilo na nekoj višoj poziciji.\\ Neke od organizacija koje objavljuju rang liste sveučilišta: Times Higher Education, Round University Ranking,
U.S. News, i Shanghai Ranking. U kontekstu ovog završnog rada, zanima nas način rangiranja Shanghai Ranking sustava. \\ Shanghai Ranking objavljuje jednom godišnje dvije 
rang liste. Prva rangira sveučilišta neovisno o područjima istraživanja i zove se Academic Ranking of World Universities (ARWU). ARWU se objavljuje od 2003. godine i  
temelji se na 6 indikatora uspjeha. Rangira više od 2000 sveučilišta, a samo najboljih 1000 objavi na službenoj stranici. Nama zanimljivija rang lista je Global Ranking of Academic Subjects
(GRAS) koja rangira sveučilišta u nekom području istraživanja. Ovu rang listu Shanghai Ranking objavljuje od 2009. godine. Zadnji ranking, 2022. godine uključivao je više od 1800 sveučilišta 
u više od 96 zemalja u 54 područja istraživanja. Fakultet elektrotehnike i računarstva u Zagrebu (FER) dio je Sveučilišta u Zagrebu te su nam rang liste u području 
računarske znanosti i inženjerstva \engl {Computer Science \& Engineering (CSE)} i elektrotehnike \engl {Electrical \& Electronic Engineering (EEE)} najzanimljivije. Problem tih rang lista 
na Shanghai Ranking stranici je taj što objavljuju samo prvih 500 najboljih sveučilišta u tim područjima što znači da ranking Sveučilišta u Zagrebu ne možemo provjeriti. 
Ovaj završni rad bavi se izradom web aplikacije koja će prikupljati podatke o sveučilištima na isti način kao i Shanghai Ranking, ali uz iznimku da nema ograničenja na 
broj sveučilišta koja se mogu pojaviti na konačnoj rang listi. Na ovaj način Sveučilište u Zagrebu, a i FER moći će pratiti svoj napredak iz godine u godinu te raditi određene 
promjene kako bi im se pozicija na rang listi poboljšala. 

\chapter{Shanghai Ranking metodologija}
Shanghai Ranking Global Ranking of Academic Subjects (GRAS) se objavljuje svake godine i temelji se na podatcima kroz četiri godine.
Ako  promatramo ranking za neku godinu $x$,
donja granica godina za podatke je $x-6$, a donja granica je $x-2$. Tako primjerice ako nas zanima ranking za 2022. godinu promatrat ćemo podatke od 2016. do 2020. godine, uključivo.

\section{Izvori za prikupljanje podataka}
Podatke za izračun vrijednosti svih indikatora osim indikatora Award prikupljamo sa baza InCites i Web of Science (WoS). 
Podatke za Award indikator prikupljamo sa raznih stranica ovisno o području koje nas zanima.
Za Computer Science \& Engineering (CSE) to je stranica A.M. Turing Award: \url{https://amturing.acm.org/}, a za Electrical \& Electronic Engineering (EEE) IEEE Awards:
\url{https://corporate-awards.ieee.org/}

\section{Minimalni broj publikacija} Kako bi sveučilište ušlo na ranking za područja Computer Science \& Engineering (CSE) i Electrical \& Electronic Engineering (EEE) mora imati minimalno 150 publikacija koje su vidljive 
na bazama Web of Science (WoS) i InCites.
\\ \section{Preslikavanje područja istraživanja}Kako bi uspješno prikupili podatke s navedenih baza moramo na tim stranicama odabrati ispravno područje istraživanja jer preslikavanja nisu $1:1$.
\\\\U sljedećim tablicama možemo vidjeti kako izgledaju preslikavanja za područja CSE i EEE.

\begin{table}[htb]
    \caption{Preslikavanje za područje Computer Science \& Engineering (CSE)}
    \label{tbl:konstante}
    \centering
    \begin{tabular}{ll} \hline
    Područje na Shanghai Ranking stranici & Područje na InCites i \\ & Web of Science (WoS) bazama\\ \hline
    Computer Science \& Engineering & Computer Science, Information Systems \\
    Computer Science \& Engineering & Computer Science, Cybernetics \\
    Computer Science \& Engineering & Computer Science, Software Engineering \\
    Computer Science \& Engineering & Computer Science, Artificial Intelligence \\
    Computer Science \& Engineering & Computer Science, Hardware \& Architecture \\
    Computer Science \& Engineering & Computer Science, Theory \& Methods \\
    Computer Science \& Engineering & Computer Science, Interdisciplinary Applications \\
    \end{tabular}
    \end{table}
    \FloatBarrier
    \hfil

\begin{table}[htb]
    \caption{Preslikavanje za područje Electrical \& Electronic Engineering (EEE)}
        \label{tbl:konstante1}
        \centering
        \begin{tabular}{ll} \hline
        Područje na Shanghai Ranking stranici & Područje na InCites i \\ & Web of Science (WoS) bazama\\ \hline
        Electrical \& Electronic Engineering &  Engineering, Electrical \& Electronic\\
        Electrical \& Electronic Engineering &  Imaging Science \& Photographic Technology\\
        \end{tabular}
        \end{table}    
        \FloatBarrier

\section{Indikatori za računanje rankinga}
Global Ranking of Academic Subjects (GRAS) se računa na temelju pet indikatora, a to su: Q1, CNCI, IC, Top i Award.
\\ \textbf{Q1} indikator predstavlja broj publikacija sveučilišta u top $25\%$ časopisa koji su izabrani za neko područje putem ankete ShanghaiRanking’s Academic Excellence Survey
(AES) tijekom relevantnog razdoblja.   
\\ \textbf{CNCI} indikator (Category Normalized Citation Impact) omjer je citiranosti objavljenih radova i prosječnih citiranosti 
radova u istoj kategoriji, iste godine i iste vrste publikacije u časopisu, od strane sveučilišta u određenom području tijekom relevantnog razdoblja.
CNCI vrijednosti manje od 1 znače da je citiranost sveučilišta manja od prosjeka, vrijednost 1 znači da je citiranost prosječna, a vrijednosti veće od
1 znače da je citiranost sveučilišta veća od prosjeka.
\\ \textbf{IC} indikator (International Collaboration) predstavlja koliko međunarodnih suradnja sveučilište u nekom području istraživanja ima.
Točna vrijednost indikatora dobije se kao omjer broja publikacija koje su pronađene u najmanje dvije različite zemlje 
u odnosu na adresu autora prema ukupnom broju publikacija iz odgovarajućeg područja istraživanja za instituciju tijekom relevantnog razdoblja.
\\ \textbf{Top} indikator predstavlja broj radova koje je sveučilište 
objavilo u Top časopisima u nekom području istraživanja tijekom relevantnog razdoblja.
Top časopise biraju profesori sveučilišta isto kao i Q1 časopise kroz anketu ShanghaiRanking’s Academic Excellence Survey (AES). 
Iznimka ovdje je područje Computer Science \& Engineering (CSE) jer se za to područje uzima broj radova predstavljenih na 31 odabranoj konferenciji, 
također kroz ShanghaiRanking’s Academic Excellence Survey (AES).
\\ \textbf{Award} indikator predstavlja broj osoba sveučilišta koje je dobilo značajnu nagradu iz nekog područja istraživanja. Značajne nagrade 
biraju se putem ShanghaiRanking’s Academic Excellence Survey (AES). Značajna nagrada za područje  Computer Science \& Engineering (CSE) je Turingova 
nagrada, a za područje Electrical \& Electronic Engineering (EEE) IEEE Medal of Honor.
Kako bi dobitak nagrade išao u korist sveučilištu, osoba koje je dobila nagradu morala je u trenutku dobitka nagrade raditi puno radno vrijeme na 
tom sveučilištu. Ako  je osoba u trenutku dobitka nagrade bila povezana s više sveučilišta ili drugih institucija, svakoj ustanovi se pridjeljuje
recipročan broj broja sveučilišta. Tako na primjer, ako je osoba bila povezana s 3 ustanove, svakoj od njih se pridjeljuje $1/3$, a ako je bila 
povezana samo s jednom ustanovom pridjeljuje se toj ustanovi $1$. 
Vrijeme dobitka nagrade također igra ulogu jer u obzir dolaze nagrade dodijeljene unazad 4 desetljeća od gornje granice godine za koju promatramo podatke.
Ako nas zanima ranking za 2022. godinu, onda je gornja granica godine za koju promatramo podatke 2020. Svakom desetljeću pridjeljuju se različite težine 
s kojima se onda množi prethodno dobiveni broj. Desetljeću najbližem sadašnjosti pridjeljuje se težina 1, a svim ostalima smanjuje se za 0.25.
Ukupna vrijednost indikatora dobije se zbrajanjem pojedinih vrijednosti za neko sveučilište u nekom području.

\begin{table}[htb]
    \caption{Primjer težina za indikator Award za ranking 2022. godine}
        \label{tbl:konstante2}
        \centering
        \begin{tabular}{cccc} \hline
        2011.-2020. & 2001.-2010. & 1991.-2000. & 1981.-1990.\\ \hline
        1&0.75&0.5&0.25\\
        \end{tabular}
        \end{table}    
        \FloatBarrier
\newpage
\section{Računanje ukupnog rezultata sveučilišta}
Jednom kada imamo sve indikatore za neko sveučilište ukupna brojčana vrijednost prema kojoj ćemo rangirati sveučilišta izračuna 
se na sljedeći način:
\\Svaki od indikatora osim CNCI podijelimo s najvećom vrijednosti indikatora od svih sveučilišta, iz tog broja izvadimo 
drugi korijen te dobiveni broj pomnožimo s težinom tog indikatora.
Na kraju dobivene vrijednosti sumiramo.
\\
\\ Izračun vrijednosti koja se pridjeljuje nekom indikatoru: 
\begin{align}
    \sqrt[\leftroot{-2}]{\frac{vrijednost \; indikatora}{max(indikator)}} \; * tezina(indikator) \label{eq:a}
\end{align}
\\ Indikator CNCI je poseban te se njemu pridjeljuje sljedeća vrijednost: \\ 
\begin{align}
    \sqrt[\leftroot{-2}]{\frac{vrijednost \; CNCI \; indikatora}{min(2*average(CNCI), max(CNCI))}} \; * tezina(CNCI) \label{eq:b}
    \end{align}
\\ Ukoliko je vrijednost CNCI indikatora veća od vrijednosti brojnika u izrazu \ref{eq:b} onda se sveučilištu automatski pridjeljuje vrijednost 100.

\begin{table}[htb]
    \caption{Težine s kojima množimo indikatore}
        \label{tbl:konstante3}
        \centering
        \begin{tabular}{ccccc} \hline
        Q1 & CNCI & IC & Top & Award \\ \hline
        100&100&20&100&100\\
        \end{tabular}
        \end{table}    
        \FloatBarrier


\subsection{Primjer izračuna vrijednosti za ranking}
Promatramo izračun vrijednosti za sveučilište University of California, Berkeley koje je 2022. godine bilo prvo na rankingu u području 
Electrical \& Electronic Engineering (EEE).
\\
\\ \underline{Vrijednost Q1:} 
\\ Pretraživanjem InCites baze vidimo da navedeno sveučilište ima vrijednost indikatora Q1 603. Najveća vrijednost Q1 indikatora u tom razdoblju iznosi 3285.
Vrijednost koja se veže za Q1 indikator iznosi;  \; $\sqrt[\leftroot{-2}]{\frac{603}{3285}} \; * 100 = 42.8$
\\
\\ \underline{Vrijednost CNCI:} 
\\ Pretraživanjem InCites baze vidimo da navedeno sveučilište ima vrijednost indikatora CNCI 1.71. Najveća vrijednost CNCI indikatora u tom razdoblju iznosi 4.13.
Kako je riječ o indikatoru CNCI u obzir još moramo uzeti i dvostruku prosječnu vrijednost svih CNCI vrijednosti, a to je 2.27.
Vrijednost koja se veže za Q1 indikator iznosi: \; \\ $\sqrt[\leftroot{-2}]{\frac{1.71}{min(2.27, 4.13)}} \; * 100 = 86.8$
\\
\\ \underline{Vrijednost IC:} 
\\ Pretraživanjem InCites baze vidimo da navedeno sveučilište ima vrijednost indikatora IC 57.18. Najveća vrijednost IC indikatora u tom razdoblju iznosi 97.15.
Vrijednost koja se veže za IC indikator iznosi: \; \\ $\sqrt[\leftroot{-2}]{\frac{57.18}{97.15}} \; * 20 = 15.3$
\\
\\ \underline{Vrijednost Top:} 
\\ Pretraživanjem InCites baze vidimo da navedeno sveučilište ima vrijednost indikatora Top 25. Najveća vrijednost Top indikatora u tom razdoblju iznosi 25.
Vrijednost koja se veže za Top indikator iznosi: \; \\ $\sqrt[\leftroot{-2}]{\frac{25}{25}} \; * 100 = 100$
\\
\\ \underline{Vrijednost Award:} 
\\ Pretraživanjem stranice IEEE Awards vidimo da je navedeno sveučilište osvojilo nagradu IEEE Medal of Honor sljedećih godina: 1985., 1995., 1998., 2020. Kako je 
svake godine nagradu osvojila uvijek jedna osoba koja nije bila vezana za nijednu drugu ustanovu, vrijednost 
indikatora Award je $1*0.25+1*0.5+1*0.5+1*1 = 2.25$ (u obzir uzimamo težine koje se vežu za godinu osvajanja nagrade). Najveća vrijednost 
indikatora Award u tom razdoblju iznosi 3. Vrijednost koja se veže za Award indikator iznosi: \; \\ $\sqrt[\leftroot{-2}]{\frac{2.25}{3}} \; * 100 = 86.6$
\\Ukupan rezultat sveučilišta University of California, Berkeley 2022. godine iznosi: $42.8+86.8+15.3+100+86.6 = 331.5$ 

\chapter{Funkcionalni zahtjevi}
\section{Funkcionalni zahtjevi}
Funkcionalni zahtjevi predstavljaju sve usluge koje programski proizvod mora pružiti korisnicima te definiraju kako sustav reagira na određene ulazne poticaje.
\\ Aktori ovog programskog sustava su korisnici, React web grafičko sučelje, Node.js poslužitelj i PostgreSQL baza podataka.
\\
\\Korisnici mogu:

a) pregledati bazu procjene rankinga sveučilišta za određenu godinu u područjima Computer Science \& Engineering (CSE) 
i Electrical \& Electronic Engineering (EEE)

b) pregledati vrijednosti svih indikatora nekog sveučilišta pomoću kojih se računa procjena rankinga za željeno područje i godinu

c) usporediti vrijednosti Shanghai Ranking sustava s procijenjenim vrijednostima za sva sveučilišta u nekom području i za neku godinu

d) pregledati uspješnost procjene rankinga za željeno područje i godinu

e) pregledati grafički prikaz promjene procjene vrijednosti indikatora i pozicije na rankingu sveučilišta u nekom području tijekom svih godina 
za koje se računa ranking, kao i pratiti napredak sveučilišta za trenutnu godinu
\\
\\React web grafičko sučelje može:

a) omogućiti korisniku odabir područja i godine za pregled procjene rankinga sveučilišta

b) omogućiti korisniku tablični prikaz procjene rankinga sveučilišta sa svim vrijednostima indikatora i pozicije sveučilišta za željeno područje i godinu

c) omogućiti korisniku pretragu sveučilišta na rankingu za određeno područje i godinu

d) omogućiti korisniku grafički prikaz procjene promjene vrijednosti indikatora i pozicije nekog sveučilišta za željenu godinu i područje

e) omogućiti korisniku prikaz uspješnosti procjene rankinga u odnosu na Shanghai Ranking sustav za željeno područje i godinu
\\
\\Node.js poslužitelj može:

a) inicijalno napuniti bazu podataka koristeći baze Web of Science (WoS) i InCites s podatcima potrebnim za izračun procjene rankinga sveučilišta

b) svakih dva tjedna prikupiti podatke sa baza Web of Science (WoS) i InCites za procjenu rankinga sveučilišta u trenutnoj godini

c) Nudi React web grafičkom sučelju krajnje točke potrebne za prikaz podataka korisniku.
\\
\\PostgreSQL baza podataka može:

a) pohranjivati vrijednosti indikatora za izračun procjene rankinga sveučilišta
\\
\begin{figure}[htb]
    
    \includegraphics[scale=0.4]{slika1.png}
    \caption{Dijagram obrasca uporabe, korisnička funkcionalnost}
    \label{fig:korisnik}
    \end{figure}
\newpage
\section{Nefunkcionalni zahtjevi}
Nefunkcionalni zahtjevi opisuju koja svojstva sustav mora imati. Ova web aplikacija mora:

a) biti robusna, otporna na pogreške i stabilna

b) omogućiti brzo i fluidno korisničko sučelje

c) dati što bolju procjenu rankinga sveučilišta u odnosu na Shanghai Ranking sustav

\chapter{Korištene tehnologije}
\section{Arhitektura sustava}
Arhitektura ovog programskog sustava sastoji se kao i mnoge web aplikacije od 3 dijela:

1. Web korisničko sučelje

2. Backend poslužitelj

3. Baza podataka

\subsection{Web korisničko sučelje}
Web korisničko sučelje ove aplikacije napravljeno je u JavaScript biblioteci React. 
\\
\\ \textbf{React} je biblioteka otvorenog koda, razvijena od strane Facebooka,
prva verzija je objavljena 2013. godine te je danas vrlo raširena i često se koristi za izradu interaktivnih i dinamičkih web korisničkih sučelja. 
React omogućava jednostavniju izradu web aplikacija uz manje kodiranja i manju složenost u odnosu na izradu web aplikacije u čistom JavaScript-u.
Jedna od prednosti React biblioteke je ta što omogućava izradu jednostraničnih web aplikacija \engl{Single Page Application (SPA)} koja radi tako 
da dinamički surađuje s preglednikom te navigiranje aplikacijom ne uzrokuje odlazak na potpuno drugačiju web stranicu već se trenutna stranica 
mijenja i prepisuje s podatcima dohvaćenih s web poslužitelja. Ovu funkcionalnost omogućava dodatak react-router-dom koji nudi komponente kao što su Link.
Navigiranjem po web aplikaciji korištenjem komponente Link mijenja se URL u pregledniku, ali web stranica zapravo ostaje ista uz promijenjen sadržaj.
React koristi virtualni DOM \engl{Document Object Model} koji prati stanja 
komponenti web stranice te kada dođe do promjene stanja, u stvarnom preglednikovom DOM-u mijenja samo one elemente koji su se promjenili. Ova 
funkcionalnost uvelike poboljšava performanse web aplikacije.
Jedna stranica u React-u sastoji se od više manjih komponenti koje se mogu dijeliti između više stranica i proizvoljno gnijezditi. Ovakvom organizacijom
postižemo dobru organizaciju koda uz mogućnosti višestrukog korištenja komponenti.
Kombiniranjem navedenih funkcionalnosti React biblioteke dobivamo fluidno korisničko sučelje bez puno učitavanja stranica \engl{reload}
\\
\\ \textbf{Axios}
\\ Axios je biblioteka koja nam omogućava jednostavno stvaranje HTTP zahtjeva na backend poslužitelj i rukovanje odgovorima.
\\
\\ \textbf{Material UI}
\\ Pisanje vlastitih komponenti u React-u od početka je korisno ako nam treba komponenta nad kojom želimo imati potpunu kontrolu te veliku prilagodljivost, ali
često nam trebaju komponente s nekom generičkom funkcionalnosti te pisanje takvih komponenti svaki put od nule nije potrebno. 
Material UI je biblioteka za React koja nudi veliki broj gotovih komponenti koje se mogu prilagođavati i uređivati prema vlastitim potrebama .
\\
\\ \textbf{Tailwind CSS}
\\ Tailwind CSS je radni okvir \engl{framework} za CSS \engl{Cascading Style Sheets} koji ima niz gotovih CSS razreda koji omogućavaju lagano postizanje 
željenog izgleda komponenti.
\\ \subsection{Backend poslužitelj}
\textbf{Node.js} 
\\ Node.js je JavaScript pokretačko okruženje \engl.{runtime environment} namijenjeno izvođenju na poslužiteljskoj strani. Pokreće se na V8 JavaScript engine-u 
te omogućuje izvršavanje JavaScript koda izvan preglednika. Koristi asinkronu arhitekturu zasnovanu na događajima \engl{asynchronous event-driven architecture}
te nudi mogućnost izrade skalabilnih web aplikacija.\\\newpage \textbf{Express.js}
\\ Express.js je radni okvir \engl{framework} Node.js-a te omogućava izradu RESTful API-ja \engl{Application Programming Interface}. Express.js nam nudi
lagano upravljanje HTTP zahtjevima i izradu krajnjih točaka \engl{endpoint} s kojima će React web aplikacija komunicirati.
\\ \\ \textbf{Puppeteer}
\\ Puppeteer je Node.js biblioteka koja nudi bogati API pomoću kojeg možemo kontrolirati Chrome i Chromium preglednike koristeći DevTools protokol.
DevTools protokol omogućuje alatima upravljanje preglednicima kao što su Chrome i Chromium. Puppeteer koristimo za izradu web scrapper-a (alat koji posjećuje web stranice i 
na njima obavlja neke radnje bez potrebe za intervencijom čovjeka) za baze InCites i Web of Science. Iako navedene baze nude API pomoću kojeg bi mogli 
dohvatiti sve podatke potrebne za ranking, on se plaća. Puppeteer alatu moramo zadati niz koraka koje treba obaviti na nekoj stranici (upisati tekst u neko polje, kliknuti na gumb, otići na drugu stranicu)
kako bi postigli željeni rezultat.
\\ \\ \textbf{Node-postgres} 
\\ Kako bi naš backend poslužitelj mogao komunicirati s bazom podataka koristimo Node-postgres. To je skup Node.js modula koji nam nude 
sučelje prema bazi podataka. Pomoću ovog proširenja možemo s backend poslužitelja raditi sve uobičajene radnje s bazom podataka (stvaranje novih tablica, 
unos podataka u tablice, dohvaćanje podataka iz tablica, brisanje podataka iz tablica i razne druge radnje).
\\ \\ \textbf{Node Cron}
\\ Node Cron je Node.js modul koji nam omogućava obavljanje nekih radnji na backend poslužitelju u nama prikladno vrijeme. Node Cron modul se u ovoj 
web aplikaciji koristi kako bismo svaka dva tjedna pokrenuli alat Puppeteer koji će prikupiti najnovije podatke za ranking sveučilišta.
\\ \subsection{Baza podataka}Ova web aplikacija koristi relacijsku bazu podataka PostgreSQL. 
\\ \section{Docker}
Kako bi ovu web aplikaciju mogli lako postaviti \engl{deploy} na neki poslužitelj te tako omogućiti svim korisnicima pristup aplikaciji 
koristi se platforma Docker. Docker je platforma otvorenog koda koja se koristi za razvoj, isporuku i pokretanje aplikacija.
Docker pakira dijelove web aplikacije u odvojene dijelove koji se zovu kontejneri \engl{containers}.
Kontejneri su zapravo Docker slike \engl{images} u izvođenju. Docker slika je lagana \engl{lightweight} komponenta koja sadrži sve što je aplikaciji 
potrebno za izvođenje (izvorni kod, pokretačko okruženje \engl{runtime}, razne alate i biblioteke). Način stvaranja docker slike se definira 
datotekom Dockerfile. Jednom kad imamo stvorenu Docker sliku, možemo je pokrenuti. Time smo dobili kontejner koji se izvršava na Docker Engine-u. 
Velika prednost Dockera je ta što je podržan na puno operacijskih sustava (Windows, Linux, MacOS i ostali) te uz pomoć samo jedne naredbe i datoteke Dockerfile
možemo dobiti pokrenutu i funkcionalnu aplikaciju. Aplikacije koje se pokreću kao kontejneri rade na jednak način na svim operacijskim sustavima zbog ugrađene virtualizacije.
Docker virtualizira operacijski sustav, a ne sklopovlje. Ovo je velika prednost u odnosu na virtualizaciju koju rade virtualni strojevi. 
Virtualni strojevi emuliraju sklopovlje i upravljanjem pomoću hipervizora omogućuju da se na istom sklopovlju izvršava više operacijskih sustava. 
Docker kontejneri dijele isti operacijski sustav te svaki predstavlja poseban proces. Možemo zaključiti da Docker kontejneri zauzimaju manje 
resursa, lakši su i brži. 

\begin{figure}[htb]
    
    \includegraphics[scale=0.35]{slika2.png}
    \caption{Usporedba virtualnih strojeva i Docker načina virtualizacije}
    \label{fig:korisnik}
    \end{figure}

U slučaju ove aplikacije imat ćemo 3 Docker slike koje će postati Docker kontejneri prilikom izvođenja. Moramo imati po jednu sliku za bazu podataka,
backend poslužitelj te web korisničko sučelje. Jednom kada smo napisali Dockerfile za sve navedene dijelove aplikacije moramo ih povezati kako 
bi sa samo jednom naredbom imali potpuno funkcionalnu aplikaciju, a to radimo u datoteci docker-compose.


        
        


\chapter{Zaključak}
Zaključak.

\bibliography{literatura}
\bibliographystyle{fer}

\begin{sazetak}
Sažetak na hrvatskom jeziku.

\kljucnerijeci{Ključne riječi, odvojene zarezima.}
\end{sazetak}

% TODO: Navedite naslov na engleskom jeziku.
\engtitle{Title}
\begin{abstract}
Abstract.

\keywords{Keywords.}
\end{abstract}

\end{document}
